\documentclass{slides-style}

\slidetitle{Темы от YADRO: TMS и RISC-V}{15.09.2022}

\begin{document}

    \begin{frame}[plain]
        \titlepage
    \end{frame}

    \section{TMS}

    \begin{frame}
        \frametitle{Test Management System}
        \begin{itemize}
            \item Веб-приложение для управления тестовыми планами и результатами тестирования
            \item Аналог TestRail, Test It, TestLink и т.п.
            \item Нужно YADRO для работы QA-отдела, первый прототип по плану уже в декабре
            \item В основном на Python + PostgreSQL
            \item Нам достанется лишь небольшая часть проекта
            \item Основные фичи: тест-кейсы с настраиваемыми полями, группировка по сьютам и проектам, тэги, приоритеты, параметризация, конфигурации тестового окружения, тест-планы, учёт времени выполнения, учёт результатов с настраиваемыми полями, интеграция с Jira и системами модульного тестирования, репортинг, REST API
        \end{itemize}
    \end{frame}

    \begin{frame}
        \frametitle{Конкретные задачи (1)}
        \framesubtitle{Пока что}
        \begin{itemize}
            \item Отчёты о результатах тестирования
            \begin{itemize}
                \item Скорее всего, отдельный сервис
                \item HTML-документ со стилем и скриптами для удобного просмотра (речь о тысячах тестов)
                \item Поддержка автотестов (см. Allure Framework)
            \end{itemize}
            \item Система уведомлений
            \begin{itemize}
                \item Скорее всего, отдельный сервис, но может оказаться плагином к сайту на Django
                \item Письма, уведомления внутри системы, Telegram
            \end{itemize}
        \end{itemize}
    \end{frame}

    \begin{frame}
        \frametitle{Конкретные задачи (2)}
        \begin{itemize}
            \item Свой фронтенд
            \begin{itemize}
                \item У YADRO есть свой UI Framework, который они нам не дадут, и будут писать фронтенд на нём
                \item Open source-версия на чём угодно, взаимодействие через REST API
                \item Надо будет написать и заглушки сервера
                \item Скорее всего, командная работа
            \end{itemize}
            \item Интеграция с Jira
            \begin{itemize}
                \item Больше фронтенд внутри самой системы
                \item Скорее всего, потребуется устройство на стажировку
            \end{itemize}
        \end{itemize}
    \end{frame}

    \section{RISC-V}

    \begin{frame}
        \frametitle{RISC-V}
        \begin{itemize}
            \item Целое направление --- инструментальная поддержка процесса разработки под RISC-V
            \item Актуально, поскольку открытая процессорная архитектура
            \item Много разного плана задач, и по мере погружения в тему будет ещё Больше
            \begin{itemize}
                \item Кажется, что вплоть до кандидатской
            \end{itemize}
            \item Направления:
            \begin{itemize}
                \item Симуляция, Instruction-accurate и Cycle-accurate
                \item Оптимизация библиотек и инструментов под RISC-V
                \begin{itemize}
                    \item Рантаймы языков программирования: Python, Go, Lua
                \end{itemize}
                \item Отладка и анализ трасс
                \item IDE
            \end{itemize}
            \item Нужно желание копаться в чужом коде и непонятных штуках
            \begin{itemize}
                \item Это интересно, и уникальные знания
            \end{itemize}
        \end{itemize}
    \end{frame}

    \begin{frame}
        \frametitle{Конкретные задачи (1)}
        \framesubtitle{Пока что}
        \begin{itemize}
            \item Verilator: SCR1
            \begin{itemize}
                \item SCR1 --- открытое ядро Syntacore RISC-V
                \item Verilator --- тул, который по описанию процессора на Verilog генерирует для него симулятор
                \item Надо запустить SCR1 на Verilator и прогнать на получившемся симуляторе тесты от Syntacore
                \item Просто? Не думаю
            \end{itemize}
            \item Сравнительный обзор симуляторов
            \begin{itemize}
                \item Посмотреть на разные симуляторы, способные симулировать RISC-V
                \item Научиться их запускать
                \item Составить таблицу поддержки профилей
                \item Сравнить производительность
            \end{itemize}
        \end{itemize}
    \end{frame}

    \begin{frame}
        \frametitle{Конкретные задачи (2)}
        \begin{itemize}
            \item Оптимизация Verilator
            \begin{itemize}
                \item Известно, что некоторый код на Verilog порождает медленные симуляторы
                \item Например, сложение может быть расписано как страшная конструкция из транзисторов и Verilator не догадается заменить его на ``+''
                \item Задача --- найти такие случаи и предложить изменения в Verilator
                \item Много нетривиального кода на C++
            \end{itemize}
        \end{itemize}
    \end{frame}

    \begin{frame}
        \frametitle{Ещё направления}
        \begin{itemize}
            \item IDE
            \begin{itemize}
                \item Готовый инструментарий разработки под RISC-V с симулятором, отладкой изнутри IDE и т.п.
                \item Рекомендации по портированию --- например, подсветка инструкций с плавающей точкой, если их нет в профиле
            \end{itemize}
            \item Анализ трасс в Verilator --- печатать состояние процессора
            \item Cycle-accurate-симуляция в симуляторе SAIL для RISC V
            \begin{itemize}
                \item Почитать про это статьи, повторить эксперимент Coyote
                \item Перенести в SAIL
                \item Код на OCaml!
            \end{itemize}
            \item Оптимизация Linux под RISC-V
            \item JIT для SAIL
            \item Реализация расширений в SAIL и SPIKE
            \item Моделирование DDR
            \item ...
        \end{itemize}
    \end{frame}

\end{document}
