\documentclass[xetex,mathserif,serif]{beamer}
\usepackage{polyglossia}
\usepackage{minted}
\usepackage{tabu}

\usepackage{textpos}
\setlength{\TPHorizModule}{1cm}
\setlength{\TPVertModule}{1cm}

\useoutertheme{infolines}

\usepackage{fontspec}
\setmainfont{FreeSans}
\newfontfamily{\russianfonttt}{FreeSans}

\setbeamertemplate{blocks}[rounded][shadow=false]
\setbeamercolor*{block title example}{fg=green!50!black,bg=green!20}
\setbeamercolor*{block body example}{fg=black,bg=green!10}

\setbeamercolor*{block title alerted}{fg=red!50!black,bg=red!20}
\setbeamercolor*{block body alerted}{fg=black,bg=red!10}

\definecolor{cadmiumgreen}{rgb}{0.0, 0.42, 0.24}

\tabulinesep=0.7mm

\newcommand{\attribution}[1] {
\vspace{-5mm}\begin{flushright}\begin{scriptsize}\textcolor{gray}{\textcopyright\, #1}\end{scriptsize}\end{flushright}
}

\title{Курс ``Программирование''}
\author[Юрий Литвинов]{Ю.В. Литвинов \newline 
    \textcolor{gray}{\small\texttt{y.litvinov@spbu.ru}}
}

\date{}

\begin{document}

    \begin{frame}
        \frametitle{Цель, общая структура}
        Цель:
        \begin{itemize}
            \item После первого курса --- участвовать в летних школах, быстро стартовать проект в начале второго курса
            \item После второго курса --- быть полезными на стажировках, иметь кругозор, быть в целом готовыми на Junior-разработчика
        \end{itemize}
        Общая структура:
        \begin{itemize}
            \item 1 семестр --- алгоритмы и структуры данных на C
            \item 2 семестр --- ООП на C\#
            \item 3 семестр --- ``продвинутое'' программирование на C\#
            \item 4 семестр --- функциональное программирование на F\#
        \end{itemize}
    \end{frame}

    \begin{frame}
        \frametitle{Особенности реализации}
        \begin{itemize}
            \item Ориентированность на промышленную разработку
            \item Не требуем предварительных знаний, но ожидаем умение самостоятельно искать и обобщать информацию
            \item В основном лекционные пары, на первом курсе также практика в аудитории
            \item Много домашних работ, проверка и общение в основном удалённо
            \item Своя LMS, \url{https://hwproj.ru/}
            \begin{itemize}
                \item Нет автоматической проверки --- ценим архитектуру и качество кода выше работоспособности, код всех решений просматривается вручную
                \item Не отрицаем ценность автоматизации для базовых проверок, но пока руки не дошли
            \end{itemize}
            \item Связанные курсы: 
            \begin{itemize}
                \item Информатика --- кругозор, общая теория, но все важные вещи мы разбираем ещё раз сами
                \item Дискретная математика --- математические основы, основные алгоритмы; поддерживаем курс задачами, ещё раз обсуждаем теорию
            \end{itemize}
        \end{itemize}
    \end{frame}

    \begin{frame}
        \frametitle{1 семестр}
        \begin{scriptsize}
            \begin{itemize}
                \item Раздел 1, базовая алгоритмика
                \begin{itemize}
                    \item \scriptsize{Введение в C}
                    \item Сложность алгоритмов, стайлгайд, процесс компиляции, тестирование и отладка
                    \item Сортировки, git
                    \item Внутреннее представление данных. Файлы, структуры, указатели, .h/.c
                    \item О разработке программных продуктов. Контрольная
                \end{itemize}
                \item Раздел 2, ``динамические'' структуры данных
                \begin{scriptsize}
                    \begin{itemize}
                        \item \scriptsize{Стеки, очереди, списки}
                        \item Понятие абстрактного типа данных, ещё списки
                        \item Деревья вообще и деревья поиска
                        \item Самобалансирующиеся деревья: АВЛ, красно-чёрные, splay
                        \item Хеш-таблицы
                        \item Графы
                        \item Обзор парадигм программирования. Контрольная.
                        \item Доклады. Ещё про парадигмы программирования.
                        \item Автоматы, лексический анализ
                        \item Зачёт
                    \end{itemize}
                \end{scriptsize}
            \end{itemize}
        \end{scriptsize}
    \end{frame}

    \begin{frame}
        \frametitle{2 семестр}
        \begin{scriptsize}
            \begin{itemize}
                \item Введение в C\#
                \item ООП вообще и в C\#
                \item Модульные тесты
                \item Исключения и обработка ошибок
                \item CI, инструменты разработки, немного про лицензии
                \item Событийно-ориентированное программирование
                \item Пользовательские интерфейсы (WinForms)
                \item Контейнеры и генерики
                \item Контрольная
                \item Визуальное моделирование
                \item SOLID и общие рекомендации про хороший ООП-код
                \item Доклады
                \item Зачёт
            \end{itemize}
        \end{scriptsize}
    \end{frame}

    \begin{frame}
        \frametitle{3 семестр}
        \begin{scriptsize}
            \begin{itemize}
                \item Многопоточное программирование
                \begin{itemize}
                    \item \scriptsize{Многопоточное программирование ``на низком уровне'': поддержка со стороны ОС, планировщик, Thread, гонки}
                    \item Примитивы синхронизации, кратко про модель памяти
                    \item Практика, ``Обедающие философы''
                    \item Высокоуровневая многопоточность: пул потоков, async/await
                \end{itemize}
                \item Сетевое программирование
                \begin{itemize}
                    \item \scriptsize{Работа с сетью, низкий уровень (модель OSI, сокеты, консольные утилиты, клиент-сервер на .NET)}
                    \item Работа с сетью, высокий уровень (HTTP вообще и в .NET, REST-сервисы, безопасность)
                    \item Практика, клиент для ВКонтакте
                \end{itemize}
                \item Рефлексия
                \item Контрольная
                \item Базы данных (вообще реляционные СУБД, SQL, кратко EF Core)
                \item Веб-программирование
                \item Практика, приложение для регистрации на конференцию
                \item Контейнеризация и развёртывание, практика по Docker
            \end{itemize}
        \end{scriptsize}
    \end{frame}

    \begin{frame}
        \frametitle{4 семестр}
        \begin{scriptsize}
            \begin{itemize}
                \item Функциональное программирование, введение в F\#
                \item Языковые особенности F\#
                \item Нетипизированное $\lambda$-исчисление
                \item Генерики в F\#
                \item Объектно-ориентированное программирование в F\#
                \item Вычислительные выражения в F\#
                \item Многопоточное программирование в F\#, событийное программирование, акторная модель
                \item Синтаксический анализ на F\#, часть 1 (синтаксический анализ вообще)
                \item Синтаксический анализ на F\#, часть 2 (FParsec vs FsLex/FsYacc)
                \item Доклады
            \end{itemize}
        \end{scriptsize}
    \end{frame}

    \begin{frame}
        \frametitle{Материалы}
        \begin{scriptsize}
            \begin{itemize}
                \item Первый семестр: \url{https://github.com/yurii-litvinov/courses/tree/master/programming-1st-semester}
                \item Второй семестр: \url{https://github.com/yurii-litvinov/courses/tree/master/programming-2nd-semester}
                \item Третий семестр: \url{https://github.com/yurii-litvinov/courses/tree/master/programming-3rd-semester}
                \item Четвёртый семестр: \url{https://github.com/yurii-litvinov/courses/tree/master/structures-and-algorithms}
            \end{itemize}
        \end{scriptsize}
        Материалы включают в себя TeX-овские исходники презентаций, конспекты (кроме первого семестра, там занятия в более интерактивном формате), условия домашних задач.
    \end{frame}

\end{document}

