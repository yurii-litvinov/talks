\documentclass{slides-style}

\slidetitleext{Образовательные программы\\ \enquote{Технологии программирования} и \enquote{Программная инженерия}}{30.03.2025}{Техпрог и ПИ}

\begin{document}

    \begin{frame}[plain]
        \titlepage
    \end{frame}

    \section{Технологии программирования}

    \begin{frame}{Технологии программирования}
        \subtitle{Основная информация}
        \begin{center}
            \begin{tabular}{p{5cm} p{5cm}}
                Срок обучения           & \emph{4 года} \\
                \\
                Форма обучения          & \emph{очная} \\
                \\
                Вступительные испытания & Математика, \newline Информатика, \newline Русский язык \\
                \\
                Количество мест         & 55 бюджет, 6 договор\\
                \\
                Конкурс                 & 20-25 заявлений на место
                \\
                Проходной на бюджет     & \emph{Порядка 255}
            \end{tabular}
        \end{center}
    \end{frame}

    \begin{frame}{Преимущества програмы}
        \begin{itemize}
            \item Сбалансированное соотношение математики и программирования
            \item Возможность выбора индивидуальной траектории обучения
            \begin{itemize}
                \item Четыре выпускающие кафедры: информатики, информационно-аналитических систем, параллельных алгоритмов, системного программирования
                \item Огромный выбор элективов с 3-го курса
                \item Несколько траекторий по программированию на 1-2 курсах
            \end{itemize}
            \item Прочные связи с ведущими IT-компаниями
            \item Выпускники программы среди предпринимателей, директоров, ведущих специалистов
            \item Проверено временем: диплом, знания и связи
            \item Интересный процесс обучения, доступны все возможности
        \end{itemize}
    \end{frame}

    \begin{frame}{Тематика основных курсов}
        \begin{itemize}
            \item Продвинутая теория и практика по информатике
            \item Разработка компиляторов
            \item Архитектура процессоров и программного обеспечения
            \item Работа с большими данными
            \item Вероятностные модели, вычислительная математика
            \item Доказательство корректности программ
            \item Функциональные языки программирования
            \item Параллельные и распределённые системы
            \item Управление проектами
            \item \highlight{И многое другое!}
        \end{itemize}

        \vspace{3mm}

        \emph{У нас почти не учат машинному обучению!}
    \end{frame}

    \section{Программная инженерия}

    \begin{frame}{Программная инженерия}
        \subtitle{Основная информация}
        \begin{center}
            \begin{tabular}{p{5cm} p{5cm}}
                Срок обучения           & \emph{4 года} \\
                \\
                Форма обучения          & \emph{очная} \\
                \\
                Вступительные испытания & Математика, \newline Информатика, \newline Русский язык \\
                \\
                Количество мест         & 45 бюджет, 12 договор \\
                \\
                Конкурс                 & 55-90 заявлений на место
                \\
                Проходной на бюджет     & \emph{Порядка 275}
            \end{tabular}
        \end{center}
    \end{frame}

    \begin{frame}{Преимущества програмы}
        \begin{itemize}
            \item Много программирования и программно-инженерных дисциплин, таких как управление проектами
            \item Одна выпускающая кафедра: системного программирования
            \item Прочные связи с ведущими IT-компаниями, выпускники программы повсюду
            \item Выдающийся коллектив преподавателей с опытом работы в индустрии
        \end{itemize}
    \end{frame}

    \begin{frame}{Тематика основных курсов}
        \begin{itemize}
            \item дискретная математика, математическая логика, вычислительная математика, теория вероятностей и математическая статистика...
            \item Теория и практика по программированию, алгоритмы и структуры данных, анализ алгоритмов
            \item Программная инженерия, управление проектами
            \item Архитектура ЭВМ, базы данных, теория формальных языков и компиляции, архитектура ПО
            \item Параллельные и распределённые системы
            \item Компьютерная графика
            \item Методы анализа информации, информационный поиск, компьютерная безопасность
            \item Возможность выбрать из четырёх иностранных языков
            \item \highlight{И многое другое!}
        \end{itemize}

        \vspace{3mm}

        \emph{Тут совсем не учат машинному обучению!}
    \end{frame}

    \section{Различия}

    \renewcommand{\arraystretch}{1.5}

    \begin{frame}{Программная инженерия\newline vs Технологии программирования}
        \begin{small}
            \begin{center}
                \begin{tabular}{p{3cm} p{4cm} p{4cm}}
                                                        & \highlight{Технологии\newline программирования} & \highlight{Программная\newline инженерия} \\
                    Вариативность программы             & Четыре кафедры, самые разные курсы и задачи & Фокус на системном программировании и программной инженерии \\
                    Математическая подготовка           & Сильная              & Достаточно сильная \\
                    Набор                               & 55 мест              & 45 мест \\
                    Конкурс                             & 20-25 на место       & 55-90 на место \\
                    \emph{Возможность\newline халявить} & Да                   & Нет
                \end{tabular}
            \end{center}
        \end{small}
    \end{frame}

    \section{Общее}

    \begin{frame}{Как учим?}
        \begin{itemize}
            \item Практика программирования на младших курсах от профессионалов
            \begin{itemize}
                \item Упор на индустриальные практики
            \end{itemize}
            \item Летние школы
            \item Стажировки
            \item Совместные с индустриальными компаниями лаборатории, исследовательские проекты
            \item Большой упор на учебные практики и самостоятельную работу
            \begin{itemize}
                \item Большинство практик --- со специалистами из крупных компаний в качестве консультантов
                \item Защиты с потенциальными работодателями в комиссиях
            \end{itemize}
        \end{itemize}
    \end{frame}

    \begin{frame}{Пример стажировки, куда проходят наши студенты}
        \begin{itemize}
            \item YADRO ИМПУЛЬС: \url{https://edu.yadro.com/impulse/}
            \item Оплачиваемая стажировка, 2 месяца, 40 часов в неделю
            \begin{itemize}
                \item Летом после 2-го или 3-го курса
            \end{itemize}
            \item Возможность продолжить работу в компании (88\% стажёров в 2024-м году)
            \item Направления: C++, Go, Python, Rust, Android (Java, Kotlin), C, системное программирование, JavaScript (React, Angular), Администрирование Linux, Bash
            \item Также: DevOps, Информационная безопасность
            \item Мы в целом всему этому учим. Либо учим так, что без проблем сможете сами освоить
        \end{itemize}
    \end{frame}

    \begin{frame}{Что ещё у нас есть}
        \begin{itemize}
            \item Участие в научных конференциях
            \item Спонсорские стипендии
            \item Хорошие общежития для иногородних в пяти минутах от факультета
            \item Кружки по спортивному программированию, языкам программирования, проектам, компьютерной безопасности и~т.д.
            \item Военно-учебный центр
            \item Студсовет, активное участие студентов в жизни факультета
            \item Очень много внеучебной деятельности
            \begin{itemize}
                \item Неделя матмеха!
            \end{itemize}
        \end{itemize}
    \end{frame}

    \begin{frame}{Где работают наши выпускники}
        Собственные компании, научная карьера в вузах или в РАН \\
        Иначе:

        \vspace{3mm}

        \begin{tabular}{p{6cm} p{6cm}}
            \highlight{Отечественные IT-компании:} & \highlight{Зарубежные IT-компании:} \\
            YADRO                                  & Google \\
            Яндекс                                 & Microsoft \\
            Т-Банк                                 & JetBrains \\
            Ланит
        \end{tabular}

        \vspace{3mm}

        И т.д., список очень длинный

        \vspace{3mm}

        Зарплаты часто больше \highlight{200К} рублей в месяц через год после выпуска
    \end{frame}

    \begin{frame}
        \begin{Huge}
            Учиться сложнее, чем поступить
        \end{Huge}

        \vspace{1cm}

        \highlight{Но это того стоит!}
    \end{frame}

    \begin{frame}{Где подробнее узнать о поступлении}
        \begin{columns}
            \begin{column}{0.5\textwidth}
                \begin{itemize}
                    \item Сайт приёмной комиссии СПбГУ: \url{https://abiturient.spbu.ru/}
                    \item Сайт кафедры системного программирования: \url{https://se.math.spbu.ru/}
                \end{itemize}
            \end{column}
            \begin{column}{0.5\textwidth}
                \begin{center}
                    \includegraphics[width=0.8\textwidth]{sysprogAdmission.png}
                    
                    Telegram-чат для поступающих:
                    \url{https://t.me/sysprog_admission}
                \end{center}
            \end{column}
        \end{columns}
    \end{frame}

\end{document}
